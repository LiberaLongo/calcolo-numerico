\documentclass{article}

\title{FORMULARIO CALCOLO NUMERICO}
\date{2023-01-27}
\author{Libera Longo}

%link utili:
% https://latex-tutorial.com/tutorials/

% https://latex.codecogs.com/eqneditor/editor.php

% https://www.overleaf.com/learn/latex/Lists

% https://wwwcdf.pd.infn.it/AppuntiLinux/latex_ambienti_matematici.htm

\usepackage[a4paper,left=2cm,right=3cm,top=2.5cm,bottom=2.5cm]{geometry}	%margins
\usepackage{imakeidx}	%indice
\usepackage{xcolor}	%colore testo
\usepackage{amsmath}	%matematica formule
\usepackage{amsfonts}
\usepackage{parskip}	%paragraph with \\

%\makeindex

\begin{document}
	\maketitle
	%\printindex
	\section{Floating Point}

Si definisce \textcolor{red}{insieme dei numeri macchina (floating-point)} con \textit{t} cifre significative,
base $\beta$ e range $( L, U )$, l'insieme dei numeri reali definito nel modo seguente
\[
	\mathbb{F} ( \beta , t, L, U ) = \{ 0  \} \cup \left \{ x \in \mathbb{R} = sign(x) \beta^p \sum_{i=1}^{t} d_{i} \beta^{-i} \right \}
\] \\
ove $t, \beta$ sono interi positivi con {\color{blue} $ \beta \geq 2 $ }.
Si ha inoltre
\begin{center}
	{\color{blue} $ 0 \leq d_i \leq \beta -1, \;\;\;\;\; i = 1, 2,... $} \\
	{\color{blue} $ d_i \neq 0, \;\;\;\;\; L \leq p \leq U $ } $ \;\;\;\;\; p \in \left [ L, U \right ] $
\end{center}
Usualmente $U$ è positivo e $L$ negativo. \\
\textcolor{red}{I numeri dell'insieme $\mathbb{F}$ sono ugualmente spaziati tra le successive potenze di $\beta$, ma non su tutto l'intervallo.} \\
Esempio
$\beta = 2, t = 3, L = -1, U = 2 $ \\
$\mathbb{F} = \{ 0 \} \cup \{ 0.100 \times 2^p, \; 0.101 \times 2^p, \; 0.110 \times 2^p, \; 0.111 \times 2^p , \; p = -1, 0, 1, 2 \} $
dove $0.100 \; 0.101 \; 0.110 \; 0.111$ sono tutte le possibili mantisse e $p$ il valore dell'esponente.

\noindent\rule{\textwidth}{0.4pt}

\begin{itemize}
	\item In rappresentazione posizionale un numero macchina $x \neq 0$ viene denotato con $x = \pm .d_1 d_2 ... d_t \beta ^p$
	\item La maggior parte dei calcolatori ha la possibilità di operare con lunghezze diverse di $t$, a cui corrispondono, ad esempio, la semplice e la doppia precisione.
	\item E' importante osservare che l'insieme $\mathbb{F}$ non è un insimee continuo e neppure infinito.
\end{itemize}
Come rappresentare un numero reale positivo $x$ in un sistema di numeri macchina $\mathbb{F} ( \beta, t, L, U)$ ?
\begin{itemize}
	\item Il numero $x$ è tale che $L \leq p \leq U$ e $d_i = 0$ per $i > t$; allora $x$ è un numero macchina ed è rappresentato esattamente ({\color{blue} $x \in \mathbb{F}$}).
	\item $p \notin \left [ L, U \right ]$; il numero non può essere rappresentato esattamente ({\color{blue} $x \notin \mathbb{F}$}).
\\Se $p < L$, si dice che si verifica un underflow; solitamente si assume come valore approssimato del numero x il numero zero.
\\Se $p > U$ si verifica un overflow e solitamente non si effettua nessuna approssimazione, ma il sistema di calcolo dà un avvertimento più drastico, come ad esempio, l'arresto del calcolo.
\end{itemize}

\noindent\rule{\textwidth}{0.4pt}

Se una matrice $A n \times n$ ha un autovettore $\lambda = 0$, allora A è singolare.\\
Il costo computazionale per la risoluzione di un sistema triangolare è di: 
$O ( \frac{n^2}{2} ) $
	\newpage

	\section{Condizionamento e Stabilità}
	\section{Fattorizzaizone LR o LU}
	\subsection{Fattorizzazione LU con pivot}
	\section{Fattorizzazione di Cholesky}
	\section{Interpolazione}
	\section{Chebyshev}
	\section{Norme}
	\section{Punti di massimo e minimo}
	\section{direzione e metodi di discesa}
	\section{minimi quadrati}
\end{document}


